\documentclass[12pt]{scrartcl}



\setlength{\parindent}{0pt}
\setlength{\parskip}{.25cm}

\usepackage{graphicx}

\usepackage{xcolor}

\definecolor{darkred}{rgb}{0.5,0,0}
\definecolor{darkgreen}{rgb}{0,0.5,0}
\usepackage{hyperref}
\hypersetup{
  letterpaper,
  colorlinks,
  linkcolor=red,
  citecolor=darkgreen,
  menucolor=darkred,
  urlcolor=blue,
  pdfpagemode=none,
  pdftitle={CS1 - Lab 14.0 - Java},
  pdfauthor={Christopher M. Bourke},
  pdfkeywords={}
}

\definecolor{MyDarkBlue}{rgb}{0,0.08,0.45}
\definecolor{MyDarkRed}{rgb}{0.45,0.08,0}
\definecolor{MyDarkGreen}{rgb}{0.08,0.45,0.08}

\definecolor{mintedBackground}{rgb}{0.95,0.95,0.95}
\definecolor{mintedInlineBackground}{rgb}{.90,.90,1}

%\usepackage{newfloat}
\usepackage[newfloat=true]{minted}
\setminted{mathescape,
               linenos,
               autogobble,
               frame=none,
               framesep=2mm,
               framerule=0.4pt,
               %label=foo,
               xleftmargin=2em,
               xrightmargin=0em,
               startinline=true,  %PHP only, allow it to omit the PHP Tags *** with this option, variables using dollar sign in comments are treated as latex math
               numbersep=10pt, %gap between line numbers and start of line
               style=default, %syntax highlighting style, default is "default"
               			    %gallery: http://help.farbox.com/pygments.html
			    	    %list available: pygmentize -L styles
               bgcolor=mintedBackground} %prevents breaking across pages
               
\setmintedinline{bgcolor={mintedBackground}}
\setminted[text]{bgcolor={mintedBackground},linenos=false,autogobble,xleftmargin=1em}
%\setminted[php]{bgcolor=mintedBackgroundPHP} %startinline=True}
\SetupFloatingEnvironment{listing}{name=Code Sample}
\SetupFloatingEnvironment{listing}{listname=List of Code Samples}

\title{CSCE 155 - Java}
\subtitle{Lab 14.0 - Graphical User Interface Programming}
\author{~}
\date{~}

\begin{document}

\maketitle

\section*{Prior to Lab}

Before attending this lab:
\begin{enumerate}
  \item Read and familiarize yourself with this handout.
\end{enumerate}

Some additional resources that may help with this lab:
\begin{itemize}
  \item Oracle's Java Swing Tutorials:\\
	\url{http://docs.oracle.com/javase/tutorial/uiswing/}
  \item Another Swing tutorial:
	\url{http://zetcode.com/tutorials/javaswingtutorial/}
\end{itemize}

\section*{Peer Programming Pair-Up}

To encourage collaboration and a team environment, labs will be
structured in a \emph{pair programming} setup.  At the start of
each lab, you will be randomly paired up with another student 
(conflicts such as absences will be dealt with by the lab instructor).
One of you will be designated the \emph{driver} and the other
the \emph{navigator}.  

The navigator will be responsible for reading the instructions and
telling the driver what to do next.  The driver will be in charge of the
keyboard and workstation.  Both driver and navigator are responsible
for suggesting fixes and solutions together.  Neither the navigator
nor the driver is ``in charge.''  Beyond your immediate pairing, you
are encouraged to help and interact and with other pairs in the lab.

Each week you should alternate: if you were a driver last week, 
be a navigator next, etc.  Resolve any issues (you were both drivers
last week) within your pair.  Ask the lab instructor to resolve issues
only when you cannot come to a consensus.  

Because of the peer programming setup of labs, it is absolutely 
essential that you complete any pre-lab activities and familiarize
yourself with the handouts prior to coming to lab.  Failure to do
so will negatively impact your ability to collaborate and work with 
others which may mean that you will not be able to complete the
lab.  

\section{Lab Objectives \& Topics}
At the end of this lab you should be familiar with the following
\begin{itemize}
  \item Graphical User Interface and Event-based Programming in Java using Swing
  \item Compiling and running a GUI program
\end{itemize}

\section{Background}

Many programs interact with users using a Graphical User 
Interface (GUI).  Traditional GUI design involves the creation 
and interaction of widgets: general graphical elements that 
include labels, text boxes, buttons, etc.  Most GUI programming 
is done using an Application Programmer Interface (API) 
which is a library or framework that provides a lot of the core 
functionality including:
\begin{itemize}
  \item A window manager that handles the interaction of widgets
  \item A windowing system that works with the underlying operating 
  	system and hardware to render the graphics
  \item Factory functions that can be used to construct and configure widgets
\end{itemize}
The particular API that we will work with is Java?s Swing API which 
is part of Java?s Standard Developer Kit (SDK).  Swing provides 
functionality for creating and managing widgets as well as the 
ability to change the look and feel of the application.  Because 
it is Java, an application written in Swing will operate across 
different operating systems and should provide the same user experience.

\subsection*{A simple GUI program}

We have provided a simple calculator program, (in the class file 
\mintinline{text}{Calculator.java}) that simulates a simple calculator.  
Two editable input boxes and one non-editable output box have 
been implemented along with several buttons to support arithmetic 
operations (addition and subtraction).  The operations work by 
grabbing the values in the two input boxes, performing the 
arithmetic operation and placing the result into the output box.

You will extend the functionality of this program by adding another button 
to support division.  To do this, you will need to add code to create and 
configure the button, create an event listener (an \mintinline{java}{ActionListener}) 
and register it with your button.  You will also need to add the button to 
the appropriate container.

\section{Activities}

Clone the project code for this lab from GitHub using the following
URL: \url{https://github.com/cbourke/CSCE155-Java-Lab14}.

\subsection{Getting Familiar with Swing}

Several additional Swing examples have been provided to you 
including the incomplete calculator program.  Familiarize yourself 
with the code by opening each source file and examining how they 
operate.
\begin{enumerate}
  \item Open the source file for the three examples, \mintinline{text}{HelloWorld.java}, 
 	 \mintinline{text}{KeyPad.java}, and \mintinline{text}{Calculator.java} source files
  \item Examine the source files and answer the questions on your worksheet 
\end{enumerate}
	 
\subsection{Modifying the Program}

You will now modify this program to support division.

\begin{enumerate}
  \item Run the \mintinline{java}{HelloWorld} class in Eclipse and 
  	look at the code to get an idea of how GUI components work
  \item Create a new button to support division:
  \begin{enumerate}
    \item In the \mintinline{java}{initializeGUI} method, declare a 
    	new \mintinline{java}{JButton} object with the appropriate symbol 
	representing division
    \item Create and add \mintinline{java}{ActionListener} objects and 
	associate them with your new button
    \item Add your button to the appropriate container and in the appropriate 
	order
  \end{enumerate}
  \item Compile your program and run it on several values to make 
  sure it works
\end{enumerate}

\section{Handin/Grader Instructions}

\begin{enumerate}
  \item Hand in your completed files:
    \begin{itemize}
    \item \mintinline{text}{Calculator.java}
    \item \mintinline{text}{worksheet.md}
  \end{itemize}
  through the webhandin (\url{https://cse-apps.unl.edu/handin}) 
  using your cse login and password.  
  \item Even if you worked with a partner, you \emph{both} should
  turn in all files.
  \item Verify your program by grading yourself through the
  webgrader (\url{https://cse.unl.edu/~cse155h/grade/}) using the
  same credentials.
  \item Recall that both expected output and your program's output
  will be displayed.  The formatting may differ slightly which is fine.
  As long as your program successfully compiles, runs and outputs 
  the \emph{same values}, it is considered correct.
\end{enumerate}


\section{Advanced Activity (optional)}

\begin{enumerate}
  \item Explore Swing further by changing the appearance of your calculator 
	program by adding borders, colors, etc.  Add additional functionality by adding 
	more buttons that compute mathematical functions.
  \item Further explore Swing by changing the look and feel of your application.  
	Read the following section of the Swing tutorial: \url{http://docs.oracle.com/javase/tutorial/uiswing/lookandfeel/index.html} 		and change the ``theme'' of your application.
\end{enumerate}

	
\end{document}
